\documentclass[12pt,a4paper,twoside]{book}

\usepackage[utf8]{inputenc}
\usepackage{graphicx}
\usepackage{calc}
\usepackage{multido}
\usepackage{fancyhdr}
\usepackage[polish]{babel}
\usepackage[T1]{fontenc}
\usepackage[left=2cm,right=2cm,top=3cm,bottom=3cm]{geometry}

\pagestyle{fancy}
\fancyhf{}
\renewcommand{\headrulewidth}{0pt}

\newcommand{\sigline}[1]{\makebox[\widthof{#1~}]{.\dotfill}\\#1}
\newcommand{\ndots}[1]{\multido{}{#1}{\dots}}

\begin{document}

\cfoot{Szczecin, 2018}

\includegraphics[scale=0.5]{logozut.jpg}
\hfill
\includegraphics[scale=1.1]{logowi.jpg}

\vspace{60pt}
\begin{center} 
	\Large{\textbf{Andrzej Strzelba}}\\
	\vspace{5pt}
	\large{nr albumu 66666}\\
	\vspace{5pt}
	\large{kierunek studiów: Informatyka}\\
	\vspace{5pt}
	\large{specjalność: Systemy Komputerowe i Oprogramowanie}\\
	\vspace{5pt}
	\textit{forma studiów: niestacjonarne}\\
	\vspace{30pt}
	\textbf{\uppercase{ANALIZA PRZESTRZENNA DANYCH PROBABILISTYCZNYCH Z WYKORZYSTANIEM FUZZY LOGIC ORAZ METOD SZTUCZNEJ INTELIGENCJI W SYSTEMACH WBUDOWANYCH}}\\
	\vspace{20pt}
	\textbf{\uppercase{SPACE ANALYZIS OF PROBABILLISTIC DATA INVOLVING FUZZY LOGIC AND ARTIFICAL INTELLIGENCE IN EMBEDDED SYSTEMS}}
	
	\vspace{40pt}
	praca dyplomowa inżynierska\\
	\vspace{10pt}
	napisana pod kierunkiem:\\
	\vspace{10pt}
	\Large{\textbf{prof. dr hab. inż. Jana Niezbędnego}}\\
	\vspace{10pt}
	\large{Katedra im. Świętego Jakuba}\\
\end{center}

\vspace{30pt}

\begin{tabular}{ll}
	Data wydania tematu pracy: & 6.6.2016\\
	Data złożenia pracy: & 7.7.2017\\	
\end{tabular}

\pagebreak
% ========== HERE STARTS NEW PAGE ==========
\pagestyle{fancy}
\fancyhf{}
\fancyhead[R]{Załącznik nr 3 do procedury dyplomowania}
\renewcommand{\headrulewidth}{0pt}

\vspace*{70pt}
\begin{center}
	\large{\textbf{\uppercase{OŚWIADCZENIE AUTORA PRACY\\ DYPLOMOWEJ}}}
\end{center}

\vspace{20pt}
\begin{flushleft}
	Oświadczam, że praca dyplomowa \textbf{inżynierska/magisterska }(podać rodzaj pracy) pn.
\end{flushleft}
\dotfill

\noindent\dotfill
\begin{center}
	\textsuperscript{\small{\textit{(temat pracy dyplomowej)}}}
\end{center}
napisana pod kierunkiem:

\noindent\dotfill
\begin{center}
	\textsuperscript{\small{\textit{(tytuł lub stopień naukowy imię i nazwisko opiekuna pracy)}}}
\end{center}

jest w całości moim samodzielnym autorskim opracowaniem sporządzonym przy wykorzystaniu wykazanej w pracy literatury przedmiotu i materiałów źródłowych. \\
Złożona w dziekanacie \textbf{Wydziału Informatyki}

\vspace{20pt}
treść  mojej pracy dyplomowej w formie elektronicznej jest zgodna z treścią w formie pisemnej/pisemnej i graficznej*.

\vspace{20pt}
Oświadczam ponadto, że złożona w dziekanacie praca dyplomowa ani jej fragmenty nie były wcześniej przedmiotem procedur procesu dyplomowania związanych z uzyskaniem tytułu zawodowego w uczelniach wyższych.

\vspace{30pt}
\begin{flushright}
	\noindent%
	\sigline{podpis dyplomanta}
\end{flushright}

\vspace{30pt}
\begin{flushleft}
	Szczecin, dn. \ndots{7}
	
	\vspace{30pt}
	* - niepotrzebne skreślić
\end{flushleft}


\end{document}